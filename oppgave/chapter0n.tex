% !TEX encoding = UTF-8 Unicode
%!TEX root = thesis.tex
% !TEX spellcheck = en-US
%%=========================================
\chapter[Conclusions]{Conclusions, Discussion, and Recommendations for Further Work}

Many attempts have been made in terms of semantic segmentation, and advances are still being made. 

%%=========================================
\section{Summary and Conclusions}
In this paper, we have been investigating different approaches to estimate the total inventory of the famous price settlement point for West Texas Intermediate of the New York Mercantile Exchange in Cushing, Oklahoma. To do so, two different problems have been identified, which is locating the oil tanks in from an aerial image and estimating the inventory based on the height of the tanks floating roofs.

First, the theoretical background for different methods is presented, including standard approaches for semantic segmentation, theory related to artificial neural networks, height estimation using SAR and InSAR analysis and trigonometrical height estimation using building shadows cast by the sun.

Furthermore, related work for the different theoretical subjects is presented. The primary purpose of this chapter is to give an understanding of what has been done so far, and where the technology is today.

Then the different methods are compared, evaluated and combined to come up with solutions that are suitable for the task at hand. Two different approaches were selected as most suitable; Direct estimation of the tanks inventory from shadow measurements using a modified reset and using a generalized Hough transform for tank detection, thus limiting the shadow search space, and then applying a modified ResNet for estimating a single tanks inventory.

Lastly, the implementation of the method is discussed by presenting some tools that enable fast experimentation with deep neural networks and different approaches related to requiring enough training data is discussed.

There exists an extensive amount of different techniques for both aerial image segmentation and height estimation from remote sensing. In the last five years, convolutional neural have played a crucial role in the development of image segmentation methods, but their applications reach far beyond this particular field. In this paper both traditional and machine learning based methods have been explored through theory and related work. Using this knowledge two methods has been evaluated as the most plausible approached to retrieve an accurate and frequently updated estimate of the total oil inventory in the tanks located in Cushing, Oklahoma.

For further work, these methods should be implemented and tested to determine if it is possible to do precise enough estimations. If so, there would lie a great economic potential in these numbers because of Cushing's importance as a price settling point for West Texas Intermediate of the New York Mercantile Exchange.

\section{Discussion}
The two methods selected as potential solutions in this paper are chosen because they present the largest likelihood of solving the specific problem presented. The decision is based on the availability of frequently updated satellite data and a certain level of accuracy. This being said, there lies a huge potential in information extraction from frequently updated remote sensing data that could benefit from the other methods presented in this paper. Especially methods within the field of machine learning could be interesting because of their ability to generalize beyond the data they have been trained on. 

With new satellite technology being developed every year providing open, updated data at increasing frequencies, it is important to develop methods that can derive, process and analyze these methods automatically. Hopefully, this article can provide some insight as to what exists and how they can be taken advantage of in any further research.