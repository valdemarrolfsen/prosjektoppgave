% !TEX encoding = UTF-8 Unicode
%!TEX root = thesis.tex
% !TEX spellcheck = en-US
%%=========================================
\chapter{Introduction}
Pattern recognition and feature extraction in remote sensing, has been a field of research for many decades. The ability to extract geospatial information directly from satellite imagery has been crucial for fields such as environmental and demographic research. A vast development within the field of machine learning and artificial intelligence the last years, has enabled many researchers to find useful applications of such algorithms within their own scientific fields.

With new satellite technology providing frequently updated imagery, with resolutions less than 0.5 meters, the amount of data that can be extracted is almost incomprehensible. What part will pattern recognition using machine learning play in the future of geospatial analyses? 

%%=========================================
\section{Background}
% In this section, you should present the problem that you are going to investigate or analyze; why this problem is of interest; what has, so far, been done to solve the problem, and which parts of the problem that remain.

Satellites has been collecting earth observation data for decades. Since the satellite Explorer 6 took the first picture of the earth in 1959, millions of satellite images has been taken, processed and stored \cite{Esa2009a}. This information however, has been very difficult to access, and even harder to analyze when accessible. However, during the last decade, the development of machine learning based methods for Earth Science applications has experiences a considerable leap forward \cite{Lary2010}.

In 2014 the first satellite in the new family of earth observation satellites, called the Sentinels, was launched from Kourou, French Guiana. Since then 7 different constellations, each consisting of two satellites,  have been launched, and are now orbiting and monitoring the earths surface. The goal of these satellites is to produce a continuous stream of timely data for Europe's Copernicus program, which will be used for environmental monitoring. The different constellations have different missions when it comes to providing datasets for the Copernicus Service. While some provide very specific data, such as monitoring the earths atmosphere, other constellations provide more general data, such as multi-spectral, high-resolution imagery of the earths surface. In order to maximize the usage of these temporal datasets, they have all been provided free of charge to the public. The fact that all of these datasets are given free of charge

World View constellation

Cushing Oklahoma

%%=========================================
\subsection*{Problem Formulation}
You should define your problem in a clear an unambiguous way and explain why this is a problem, why it is of interest---and to whom. It is also important to delimit the problem area.
%%=========================================
\subsection*{Related work}
You should here present the main books and articles that treat problems that are similar to what  you are studying, and give proper references to each of these as they are reported. If you,  later in your thesis, describe the ``state of the art'' -- with a detailed literature survey, you may just give a very brief survey here (approx. a quarter of a page). If this is the only literature survey, you need to go into more details. An objective of the literature survey is to show the reader that you are familiar with the main literature within your field of research -- so that you do not ``reinvent the wheel.''

\textbf{Sentinel-1} is a radar image mission, measuring both land and sea. It has two identical satellites orbiting 180 degrees apart.  \\
\textbf{Sentinel-2} provides multi-spectral, high resolution imagery of the earths surface. \\
\textbf{Sentinel-3} is equipped with multiple instruments used to measure sea-surface topography, sea- and land-surface temperature, ocean color and land color. \\
\textbf{Sentinel-4} \\
\textbf{Sentinel-5} \\
\textbf{Sentinel-5P} \\
\textbf{Sentinel-6} \\


References to literature can be given in two different ways:
\begin{itemize}
\item As an \emph{explicit} reference: It is shown by \citet{lundteigen08} and partly also by \citet{rausand14}  that \ldots.
\item As an \emph{implicit} reference: It is shown \citep[e.g., see][Chap. 4]{rausand04} that \ldots.
\end{itemize}
In the example above, we have used ``author-year'' references, which is the preferred format. 
\begin{remark}
Following agreement with your supervisor, you may also refer by numbers, for example,  [1]. To do this, open the file \texttt{ramsstyle.sty} and  comment out (by \%) the command \texttt{$\backslash$usepackage\{natbib\}} and un-comment the corresponding command \texttt{$\backslash$usepackage[numbers]\{natbib\}}.\footnote{Notice the strange way we have to write the ``backslash'' in the text. This is because the ``backslash'' is a command in \LaTeX.}
\end{remark}
 You may include a link to the Internet in the text or in a footnote by using a command like: \url{http://www.ntnu.edu/ross}. 

When you refer to the scientific literature, you should always write in \emph{present} tense. Example: \citet{rausand04} show that \ldots.

\begin{remark}
Hyperlinks are included by the command \texttt{$\backslash$usepackage\{hyperref}\} in \texttt{ramsstyle.sty}. If you feel that the hyperlinks are disturbing when you enter the text, or want to avoid the hyperlinks in printed text, you may either comment out or edit this command in \texttt{ramsstyle.sty}.
\end{remark}
%%=========================================
\subsection*{What Remains to be Done?}
After you have defined and delimited your problem -- and presented the relevant results found in the literature within this field, you should sum up which parts of the problem that remain to be solved.
%%=========================================
\section{Objectives}
The main objectives of this Master's project are
\begin{enumerate}
\item This is the first objective
\item This is the second objective
\item This is the third objective
\item More objectives
\end{enumerate}

The objectives shall be written as \emph{fundamental objectives} telling what to do and not \emph{means objectives} telling how to do it.

All objectives shall be stated such that we, after having read the thesis, can see whether or not you have met the objective. ``To become familiar with \ldots'' is therefore not a suitable objective.

%%=========================================
\section{Approach}
Here you should describe the (scientific) approach and experiments that you will use or have used to solve the problem and meet your objectives and tasks. Experiments may in this context relate analyses you need to carry out in order to investigate a specific hypothesis, task objective, or similar. You should specify the approach and experiments for each objective and/or task. It is preferred that you supplement your explanation of the approach with an illustration.

If there are any ethical problems related to your approach, these should be highlighted and discussed.

\section{Contributions}
Here you give a list of your main contributions in the project or master work.

%%=========================================
\section{Limitations}
In this section you describe the limitations of your study. These may be related to the study object (physical limitations, operational limitations), to the environmental and operational conditions, to the thoroughness of the analysis, and so on.

%%=========================================
\section{Outline}
Here, you give an overview of how the remaining part of the report is organized. A proposed structure of the main chapters in the report can be as follows (note that some chapters are not numbered):
\begin{itemize}
\item Preface: Contains practical information about what you have done, and where the work has been carried out. Any assumed background of the reader should be specified here.
\item Acknowledgments: Here, you show the gratitude to who have been supporting your work, professionally and family as relevant.
\item Summary: Contains the management summary, and should be a layman's explanation of what you have done and why it is important. This would be the talk you could give if you in an  interview is asked about what you did in your thesis, or if some of your relatives ask the same question. This chapter should therefore include as few domain specific words as possible, so that no detailed background in the topic is required. 
\item Chapter 1. Introduction: Structure already discussed in this chapter.
\item Chapter 2. Theoretical background: Here you identify and give the theoretical background needed in this report, with proper references to each literature reference used. The selection of what to include should be discussed and agreed with the supervisors. Theory may involve concepts, definitions, methods, regulations/key standards, theory to explain specific system behavior, and so on.
\item Chapter 3..N-2: The naming of the following chapters relies entirely on the specific topic in question. Proposed structure should be discussed with supervisor.
\item Chapter N-1 Results: This chapter should be the last chapter \textit{before} ``Conclusions, discussion, and ideas for further work''
\item Chapter N. Conclusions, discussion, and ideas for further work.
\item Bibliography
\item Appendix A etc (as needed): Appendix A may for example be acronyms as shown here.
\end{itemize}



\begin{remark}
Notice that chapter and section headings shall be written in lowercase, but that all main words should start with a capital letter.
\end{remark}



The report should be no longer than \underline{60 pages} in this format for the master as well as the specialization project, with the possible exception of appendixes (which may take up some space if including e.g. code from programming). This does not mean that the report must be at least 60 pages, and the effort should be directed to be as concise as possible throughout the report.